
Cette aventure a débuté à Polytech, lorsque ma curiosité m'a poussé à choisir
un mystérieux sujet de recherche intitulé : "\emph{Code à effacement ?}".
Aujourd'hui, j'ai compris que ce "?" représente toute l'étendue du sujet.
Difficile de réaliser le chemin parcouru jusqu'ici. Et pourtant si j'y suis
parvenu, c'est grâce aux personnes qui m'ont soutenu.

% imants
Je tiens particulièrement à remercier Imants \textsc{Svalbe} pour avoir suivi
mes travaux avec son enthousiasme légendaire, depuis ma soutenance de projet de
recherche, jusque dans son rôle de président de jury de thèse.
% duhamel, quinson
Merci également à Pierre \textsc{Duhamel} et Martin \textsc{Quinson} qui m'ont
fait l'honneur de rapporter cette thèse.
%Leurs analyses et
%remarques pertinentes ont contribué à améliorer ce manuscrit, et m'ont
%particulièrement aidé.
% lacan
Je remercie aussi chaleureusement Jérôme \textsc{Lacan} pour son accueil à
l'ISAE, et pour avoir suivi et enrichi mon parcours de son expertise des codes,
et de sa personne.

% evenou, normand et parrein
Cette thèse n'aurait pu exister sans l'aide précieuse de Pierre
\textsc{Évenou}, dont les idées et la vision m'ont beaucoup inspiré, et
surtout sans Nicolas \textsc{Normand} et Benoît \textsc{Parrein}, compères
enthousiastes et passionnés, qui ont su stimuler ma curiosité, me guider durant
ces nombreuses années, et qui ont assuré dans les hauts, comme dans les bas.
Merci pour ces moments que l'on a partagés, face au tableau, comme au bistrot.

% rozofs
Un grand merci à l'équipe de Rozo Systems, notamment à Didier \textsc{Féron},
Jean-Pierre \textsc{Monchanin} et Sylvain \textsc{David}, génies de la
programmation qui m'ont tant appris, ainsi que Louis \textsc{le Gouriellec} et
Christophe \textsc{de la Guérande} pour nos captivantes discussions.
% ivc, riton, 
Merci également à Aurore pour notre complicité, Lukáš pour nos virées
nocturnes, le man Alex pour nos galères, Josselin le stagiaire et son
yoyo, Floflo et ses fruits secs, Lulu pour sa bonne humeur, Romu pour sa
mauvaise humeur, JPeG et son remorqueur, Romain le surréaliste, et tous les
membres d'IVC pour la convivialité.

% cunche & nicta
Je n'aurais sûrement pas débuté de travail de recherche sans Mathieu
\textsc{Cunche} ni les membres du NICTA qui ont su éveiller ma curiosité pour
ce domaine.
% isae
Je pense également aux membres de l'ISAE pour leur accueil, et plus
particulièrement à Jonathan \textsc{Detchart} pour sa connaissance des codes et
de la programmation. J'ai aussi beaucoup appris grâce à nos échanges avec Suayb
\textsc{Arslan}, Andrew \textsc{Kingston} et Shekhar \textsc{Chandra}. Merci
également à José \textsc{Martinez} pour ses connaissances en \LaTeX et son aide
dans la mise en place de l'expérimentation utilisant OpenMP.

Mille pensées à mes amis, camarades de Polytech, membres du JMB et joueurs du
baby pour ces bons moments. À mes frères et mes parents pour m'avoir toujours
aidé.

% lucie
Enfin, une pensée particulière à ma charmante Lucie, pour ses relectures, son
soutien, et pour avoir fait partie des meilleurs moments durant ces trois
années.

