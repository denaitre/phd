%&these

\titre{Code à Effacement Mojette}

\soustitre{pour le Stockage Distribué}

%\titre{Approche Géométrique pour la Conception de Codes à Effacement Appliqués
%    dans le Stockage Distribué}

\title{Mojette Erasure Code}

\subtitle{for Distributed Storage}

%\title{Geometrical Approach to Design Erasure Codes for Distributed Storage}

\author{M.}{Dimitri}{Pertin}

\discipline{Informatique et applications}

\sectionCNU{27}

\institution{UN}

\doctoralschool{Sciences et technologies de l'information, et mathématiques}

\laboratory{Institut de Recherche en Communications et Cybernétique de Nantes
    (IRCCyN)}%\\
    %École polytechnique de l'université de Nantes (Polytech Nantes, membre du
    %réseau Polytech des écoles d'ingénieurs polytechniques universitaires)}

\thesisnumber{}

\date{22 avril 2016}

\reviewer{M.}
    {Pierre}
    {Duhamel}
    {Directeur de recherche CNRS}
    {L2S Centrale Supélec, Paris}

\reviewer{M.}
    {Martin}
    {Quinson}
    {Professeur des universités}
    {ENS, Rennes}

\examiner{M.}
    {Imants}
    {Svalbe}
    {Professeur étranger}
    {Monash University, Melbourne, Australie}

\examiner{M.}
    {Jérôme}
    {Lacan}
    {Professeur}
    {ISAE-SUPAERO, Toulouse}

\supervisor{M.}
    {Nicolas}
    {Normand}
    {Maitre de conférences titulaire de l'HDR}
    {Université de Nantes}
 
\cosupervisor{M.}
    {Benoît}
    {Parrein}
    {Maitre de conférences titulaire de l'HDR}
    {Université de Nantes}   

\guest{M.}
    {Évenou}
    {Pierre}
    {Ingénieur}
    {Rozo Systems}


\begin{document}

\begin{resume}
    % pas plus de 1700 caractères, espaces compris
    \footnotesize
    
Les systèmes de stockage distribués sont sujets à des défaillances inévitables
qui entraînent l'inaccessibilité temporaire, voire la perte définitive de blocs
de données. La solution classique consiste à distribuer des copies de ces
données sur différents supports de stockage, mais cela engendre un coût de
stockage important.
%
Le codage à effacement est une alternative qui permet de réduire
considérablement la quantité de redondance au prix d'une complexité
calculatoire engendrée par les opérations d'encodage et de décodage.

La transformée Mojette (une version discrète et exacte de la transformée
de \radon) est capable de représenter de manière redondante l'information,
et de reconstruire efficacement cette information. % /!\

La conception d'une version systématique du code à effacement Mojette est la
première contribution de nos travaux de recherche. Ce code a un rendement
quasi-optimal, et les algorithmes pour le mettre en œuvre sont de faible
complexité.

La seconde contribution est une nouvelle méthode distribuée pour ré-encoder de
nouveaux symboles de mots de code, sans avoir à reconstruire explicitement
l'information initiale. Cette technique permet de rétablir un niveau de
redondance voulu.

Réalisés en collaboration entre l'IRCCyN et la société Rozo Systems, ces
travaux de recherche s'intègrent dans le système de fichiers
distribué RozoFS développé par l'entreprise. En conséquence, une attention
particulière a été portée aux performances des mises en œuvres réalisées.


\end{resume}

\begin{motscles}
    Code à effacement, transformée Mojette, stockage distribué,
    tolérance aux pannes.
\end{motscles}

\begin{abstract}
    \footnotesize
    
Distributed storage systems face inevitable failures which entail temporary
unavailability, or even permanent losses of data blocks. The classical solution
is to distribute copies of this data among different storage supports, but a
significant storage cost is involved.
%
Erasure coding is an alternative that greatly reduces this amount of redundancy
at the cost of computational complexity induced by encoding and decoding
operations.

The Mojette transform (a discrete and exact version of the \radon transform) is
able to depict a redundant representation of data, and to efficiently
reconstruct it.

The design of a systematic version of the Mojette erasure code is the first
contribution of our research work. This code has an almost optimal rate, and
the devised algorithms used to implement it have a low complexity.

The second contribution is a new distributed method to re-encode new codeword
symbols, without having to explicitly reconstruct the initial information. This
technique can be used to restore a desired level of fault-tolerance.

Conducted in collaboration with IRCCyN lab and Rozo Systems Inc, our research
work is part of the distributed storage system RozoFS developed by the
company. As a consequence, particular attention has been paid to the
implementation performance.


\end{abstract}

\begin{keywords}
    Erasure code, Mojette transform, distributed storage, fault tolerance.
\end{keywords}

\maketitle

%\chapter*{Remerciements}

%\lipsum[1-2]

%\newpage

\dominitoc
\tableofcontents

\newrefsegment

%\chapter*{Introduction Générale}

%\addstarredchapter{Introduction Générale}

%\input{chapters/introduction.tex}
%\input{chapters/introduction.md}

\part{Codes à effacement en géométrie discrète}

%\input{chapters/introduction_part_1.tex}
%\input{chapters/conclusion_part_1.md}

\input{chapters/chapter01.tex}
%\input{chapters/chapter01.md}

\input{chapters/chapter02.tex}
%\input{chapters/chapter02.md}

\input{chapters/chapter03.tex}
%\input{chapters/chapter03.md}

%\input{chapters/conclusion_part_1.tex}
%\input{chapters/conclusion_part_1.md}

\part{Application au stockage distribué}

%\input{chapters/introduction_part_2.tex}
%\input{chapters/introduction_part_2.md}

\input{chapters/chapter04.tex}
%\input{chapters/chapter04.md}

\input{chapters/chapter05.tex}
%\input{chapters/chapter05.md}

\input{chapters/chapter06.tex}
%\input{chapters/chapter06.md}

%\input{chapters/conclusion_part_2.tex}
%\input{chapters/conclusion_part_2.md}

\chapter*{Conclusion Générale}

\addstarredchapter{Conclusion Générale}

%\input{chapters/conclusion.tex}
%\input{chapters/conclusion.md}

\chapter*{Perspectives}

\addstarredchapter{Perspectives}

%\input{chapters/perspectives.tex}
%\input{chapters/perspectives.md}

\endrefsegment

%\chapter*{Communications}

%\addstarredchapter{Communications}

%\input{chapters/communications.tex}
%\input{chapters/communications.md}

\printbibliography[
    heading=bibintoc,
    segment=1
]

% segment=1

% \printindex

%\section*{Nombre de ref}
%Attention : \total{citenum}\ !!! \textbackslash{}o/

\backmatter

\end{document}
