%&these

\titre{Approche Géométrique pour la Conception de Codes à Effacement Appliqués
    dans le Stockage Distribué}

\title{Geometrical Approach to Design Erasure Codes for Distributed Storage}

\author{M.}{Dimitri}{Pertin}

\discipline{Informatique et applications}

\sectionCNU{27}

\institution{UN}

\doctoralschool{Sciences et technologies de l'information, et mathématiques}

\laboratory{Institut de Recherche en Communications et Cybernétique de Nantes
    (IRCCyN)}%\\
    %École polytechnique de l'université de Nantes (Polytech Nantes, membre du
    %réseau Polytech des écoles d'ingénieurs polytechniques universitaires)}

\thesisnumber{}

\date{}

\reviewer{M.}
    {Pierre}
    {Duhamel}
    {Directeur de recherches CNRS}
    {L2S Centrale Supelec, Paris}

\reviewer{M.}
    {Martin}
    {Quinson}
    {Professeur des universités}
    {ENS, Rennes}

\examiner{M.}
    {Imants}
    {Svalbe}
    {Maître de conférences}
    {Monash University, Melbourne, Australie}

\president{M.}
    {Jérôme}
    {Lacan}
    {Professeur des universités}
    {ISAE, Toulouse}

\supervisor{M.}
    {Nicolas}
    {Normand}
    {Maitre de conférences titulaire de l'HDR}
    {Université de Nantes}
 
\cosupervisor{M.}
    {Benoît}
    {Parrein}
    {Maitre de conférences titulaire de l'HDR}
    {Université de Nantes}   

\begin{document}

\begin{resume}
    %\lipsum[1-2]
\end{resume}

\begin{motscles}
    %Code à effacement, transformée Mojette, stockage distribué.
    Robert
\end{motscles}

\begin{abstract}
    %\footnotesize
    %\lipsum[1-1]
\end{abstract}

\begin{keywords}
    Erasure code, Mojette transform, distributed storage.
\end{keywords}

\maketitle

\chapter*{Remerciements}

\lipsum[1-2]

\newpage

\dominitoc
\tableofcontents

\part{Codes à effacement en géométrie discrète}

\input{chapters/chapter01.tex}

\input{chapters/chapter02.tex}

\input{chapters/chapter03.tex}

\part{Application au stockage distribué}

\input{chapters/chapter04.tex}

\input{chapters/chapter05.tex}

\input{chapters/chapter06.tex}

\printbibliography[
]

% segment=1

% \printindex

\section*{Nombre de ref}
Attention : \total{citenum}\ !!! \textbackslash{}o/

\backmatter

\end{document}
