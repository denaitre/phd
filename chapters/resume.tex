
Les systèmes de stockage distribués sont sujets à des défaillances inévitables
qui entraînent l'inaccessibilité temporaire, voire la perte définitive de blocs
de données. La solution classique consiste à distribuer des copies de ces
données sur différents supports de stockage, mais cela engendre un coût de
stockage important.
%
Le codage à effacement est une alternative qui permet de réduire
considérablement la quantité de redondance au prix d'une complexité
calculatoire engendrée par les opérations d'encodage et de décodage.

La transformée Mojette (une version discrète et exacte de la transformée
de \radon) est capable de représenter de manière redondante l'information,
et de reconstruire efficacement cette information. % /!\

La conception d'une version systématique du code à effacement Mojette est la
première contribution de nos travaux de recherche. Ce code a un rendement
quasi-optimal, et les algorithmes pour le mettre en œuvre sont de faible
complexité.

La seconde contribution est une nouvelle méthode distribuée pour ré-encoder de
nouveaux symboles de mots de code, sans avoir à reconstruire explicitement
l'information initiale. Cette technique permet de rétablir un niveau de
redondance voulu.

Réalisés en collaboration entre l'IRCCyN et la société Rozo Systems, ces
travaux de recherche s'intègrent dans le système de fichiers
distribué RozoFS développé par l'entreprise. En conséquence, une attention
particulière a été portée aux performances des mises en œuvres réalisées.

