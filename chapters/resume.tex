Les codes à effacement permettent de générer de la redondance de données
numériques dans un système de stockage distribué. Cette redondance permet
de restaurer une partie manquante des données en cas de panne. L'avantage des
codes est de réduire considérablement la quantité de redondance générée par
rapport aux techniques classiques de réplication. Toutefois, cette réduction
s'accompagne d'une complexité calculatoire significative, pénalisant les
performances d'encodage et de décodage, ce qui limite leur utilisation aux
données froides.

Dans cette thèse, nous nous intéressons à l'utilisation de la transformation
Mojette afin de fournir un code à effacement performant, adapté aux données
chaudes. Le code qui en résulte nécessite cependant plus de redondance par
rapport aux codes classiques.

La première contribution de ces travaux de thèse traite de la conception
d'une version systématique du code à effacement Mojette. Cette version a
l'avantage d'augmenter significativement les performances du code, tout en
réduisant la quantité de redondance nécessaire.

La seconde contribution s'intéresse à l'intégration de cette solution au sein
du système de fichiers distribué RozoFS.
Cette contribution permet au système d'assurer un service continu en cas de
panne, tout en étant capable de gérer les données chaudes avec deux fois moins
de données par rapport aux systèmes basés sur la réplication.

Un troisième axe de recherche se focalise sur la conception d'une méthode
distribuée pour générer de nouveaux symboles de mots de code Mojette. Cette
technique participe à la restauration d'un seuil de redondance du système de
stockage.
