
Le codage à effacement est une technique de transmission de l'information
capable de générer de la redondance dans les informations contenues d'un
système de stockage distribué, afin de pouvoir reconstituer l'information
lorsqu'une partie de celle ci est inacessible (on parle d'effacement de la
donnée). Cette technique permet de réduire considérablement la quantité de
redondance par rapport aux techniques de réplication, au prix d'une complexité
calculatoire significative qui pénalise les opérations d'encodage et de
décodage.
%
La transformée Mojette est une transformée discrète et exacte de la transformée
de \radon. Cette technique est capable de générer une représentation redondante
de l'information et de reconstruire efficacement cette information avec une
complexité linéaire.
%
La première contribution de ces travaux correspond à la conception d'une
version systématique du code à effacement Mojette. Cette conception permet de
fournir un code au rendement quasi-optimal, et de faible complexité.
%
Réalisés en collaboration entre l'IRCCyN et la société Rozo Systems, ces
travaux de recherche s'intègrent particulièrement dans le système de fichiers
distribué RozoFS, développé par l'entreprise. Une attention particulière a
alors été portée aux performances des implémentations réalisées.
%
La seconde contribution correspond à la conception d'une nouvelle méthode
distribuée pour générer de nouveaux symboles de mots de code. Cette technique
participe à la restauration et à la flexibilité de la tolérance aux pannes du
système.

