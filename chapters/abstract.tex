
Distributed storage systems face inevitable failures which entail temporary
unavailability, or even permanent losses of data blocks. The classical solution
is to distribute copies of this data among different storage supports, but a
significant storage cost is involved.
%
Erasure coding is an alternative that greatly reduces this amount of redundancy
at the cost of computational complexity induced by encoding and decoding
operations.

The Mojette transform (a discrete and exact version of the \radon transform) is
able to depict a redundant representation of data, and to efficiently
reconstruct it.

The design of a systematic version of the Mojette erasure code is the first
contribution of our research work. This code has an almost optimal rate, and
the devised algorithms used to implement it have a low complexity.

The second contribution is a new distributed method to re-encode new codeword
symbols, without having to explicitly reconstruct the initial information. This
technique can be used to restore a desired level of fault-tolerance.

Conducted in collaboration with IRCCyN lab and Rozo Systems Inc, our research
work is part of the distributed storage system RozoFS developed by the
company. As a consequence, particular attention has been paid to the
implementation performance.

