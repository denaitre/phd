Erasure codes can generate data redundancy in distributed storage systems.
This redundancy can be used to recover missing data in case of a failure. Codes
have the benefit of reducing the generated amount of redundancy drastically,
compared to plain data replication. However, this reduction is combined with a
significant computational complexity, which penalizes encoding and decoding
performances, and limits them to cold data.

In this thesis, we focus on the use of the Mojette transform as an effective
erasure code, adapted to hot data. The resulting code requires more redundancy
than classical codes, though.

The first contribution of this research work deals with the design of a
systematic version of the Mojette erasure code. This version provides better
performances while reducing the required amount of redundancy.

The second contribution covers the integration of this solution in the
distributed file system RozoFS. This integration enables the system to provide
a continuous service despite failures, while being able to manage hot data with
half the volume of data compared to replication-based systems.

A third research focus addresses the design of a distributed method to compute
extra Mojette codeword symbols. This method contributes to restore a redundancy
threshold in the storage system.
