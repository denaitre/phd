
\begin{minipage}{15.8cm}
%\begin{minipage}{\textwidth}
\begin{tabular}{@{} l L L L L L @{} >{\kern\tabcolsep}l @{}}    \toprule
    \emph{Code} & \emph{Encoder $P$}& \emph{Encoder $Q$} & \emph{Mise à jour} &
    \emph{Décoder depuis $P$} & \emph{Décoder depuis $Q$} \\\midrule
    RS\footnote{Pour les codes de \rs, les opérations de multiplication sont
    symbolisées par $\otimes$.}  & 
        $(k-1)w$ & 
        $(k-1)w + (kw)_{\otimes}$ & 
        $3 + 1_{\otimes}$ &
        $(k-1)w$ &
        $(k-1)w + (kw)_{\otimes}$  \\ 
    \rowcolor{black!20}[0pt][0pt] \eo & 
        $(k-1)w$ &
        $(k-1)w + k-2$\footnote{\label{fn.pire_cas}Le nombre d'opérations
        correspond au pire cas (par exemple, ça peut dépendre de $\gamma(S)$
        pour \eo).} &
        $w+2$\footnoteref{fn.pire_cas} &
        $(k-1)w$ &
        $(k-1)w+2(k-2)$\footnoteref{fn.pire_cas} \\ 
    RDP &
        $(k-1)w$ &
        $(k-1)w$ &
        $4$\footnoteref{fn.pire_cas} &
        $(k-1)w$ & 
        $(k-1)w$ \\ 
    \rowcolor{black!20}[0pt][0pt]Mojette &
        $(k-1)w$ &
        $(k-1)w-k+1$ &
        $3$ &
        $(k-1) w$ &
        $\rho(k,w,l)^{(1,1)}$\footnote{Cette valeur dépend également de l'index
        $l$ de la ligne perdue, voir l'\cref{eqn.dec_sys_mojette}.}
        \\\bottomrule
\end{tabular}
\end{minipage}
